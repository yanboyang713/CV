%!TEX TS-program = xelatex
\documentclass[]{friggeri-cv}
\usepackage{afterpage}
\usepackage{hyperref}
\usepackage{color}
\usepackage{xcolor}
\hypersetup{
    pdftitle={},
    pdfauthor={},
    pdfsubject={},
    pdfkeywords={},
    colorlinks=true,       % no lik border color
    allbordercolors=white    % white border color for all
}
\addbibresource{bibliography.bib}
\RequirePackage{xcolor}
\definecolor{pblue}{HTML}{0395DE}

\begin{document}
\header{Boyang}{Yan}
     
% Fake text to add separator      
\fcolorbox{white}{gray}{\parbox{\dimexpr\textwidth-2\fboxsep-2\fboxrule}{%
.....
}}

% In the aside, each new line forces a line break
\begin{aside}
  \includegraphics[scale=0.07]{img/boyang.png}
  \section{Address}
  132 Cabbage Tree Lane,
  Fairy Meadow,  NSW,
  Australia 2500
    ~
  \section{Contact Info}
    \textbf{Tel:} +86 13821579536
    \textbf{Skype:} yanboyang
    \textbf{Wechat:} yanboyang713
    ~
  \section{Mail}
    \href{mailto:by932@uowmail.edu.au}{\textbf{by932@}\\uowmail.edu.au}
    \href{mailto:yanboyang713@gmail.com}{\textbf{yanboyang713@}\\gmail.com}
    ~
  \section{Web \& Git}
    \href{https://github.com/yanboyang713}{github.com/yanboyang713}
    \href{http://www.yanboyang.com}{yanboyang.com}
    ~
  \section{Programming}
    \includegraphics[scale=0.2]{img/programming.png}
    ~
  \section{OS Preference}
    \textbf{GNU/Linux}\includegraphics[scale=0.40]{img/5stars.png}
    \textbf{Unix}\includegraphics[scale=0.40]{img/4stars.png}
    \textbf{MacOS}\includegraphics[scale=0.40]{img/3stars.png}
    \textbf{Windows}\includegraphics[scale=0.40]{img/1stars.png}
    ~
\end{aside}

\section{Education}
\begin{entrylist}
  \entry
    {2018 - Now}
    {Master of Research (Computer Science)}
    {University of Wollongong, Australia}
    {Software Testing\\
    Main subjects: Software Testing, Operational Research, Big Data Analysis, Natural Language Processing\\}
  \entry
    {2013 - 2017}
    {Bachelor of Computer Science​}
    {University of Wollongong, Australia}
    {Major in Software Engineering\\
    Main subjects: C++ Programming, Algorithms and Data Structures, Database Systems, System Security, Multimedia Computing, Operation System, Interactive Computer Graphics, W3 Technologies, Software Development Methods \& Tools, Software Process Management, Mathematics and Statistics, Information Systems\\
    \emph{Title of the Final Project: "Human Research Management System base on Web".}\\}
  \entry
    {2013}
    {English for Tertiary Studies}
    {University of Wollongong College, Australia}
    {Main subjects: Academic English}
  \entry
    {2011 - 2012}
    {Bachelor of Business Management}
    {China University Of Mining And Technology}
    
  \entry
    {2008 - 2011}
    {High School Degree}
    {Tianjin Jingu Experimental, China}
    {Most of the courses got High Distinction}
    
  \entry
    {2005 - 2008}
    {Junior High School Degree, China}
    {Tianjin Jingu Experimental, China}
    
  \entry
    {2003 - 2005}
    {Primary School Degree}
    {Primary School attached to Tianjin Normal University, China} 
    
  \entry
    {1999 - 2003}
    {Primary School Degree}
    {Primary School attached to Xinjiang Normal University, China}  
    
\end{entrylist}

\section{Certifications}
\begin{entrylist}
  \entry
    {2017}
    {The best final project prize}
    {University of Wollongong, Australia}

  \entry
    {2017}
    {Amateur Radio Operator's certificate of proficiency (Standard)\\}
    {The Wireless Institute of Australia (WIA)}

  \entry
    {2016}
    {Ross a. Hull memorial Vhf-Uhf contest\\}
    {The Wireless Institute of Australia (WIA)}
    {Twelfth Place in Section A (Analog Modes, Best
7 Days) and Twelfth Place in Section C (Analog Modes, Best 2 Days)}

  \entry
    {2014}
    {Amateur Radio Operator's certificate of proficiency (Foundation)\\}
    {The Wireless Institute of Australia (WIA)}
   
  \entry
    {2010}
    {The Second Place Prize Awarded, Tianjin School Sports Competition Award "92" National Games Tianjin\\}
    {Tianjin Basketball Tryouts, China}
    {High school men's Group B. Primary and Secondary (Vocational) School Basketball Games. Have got the second grading certificate and title}

\end{entrylist}

\section{Training Courses}
\begin{entrylist}
  \entry
    {2012}
    {Intel innovation in EDUCATION}
    {Intel Learn Program (Technical and Community)}

\end{entrylist}

\section{Scholarships}
\begin{entrylist}
  \entry
    {2014}
    {Undergraduate Excellence Scholarship}
    {University of Wollongong, Australia}
    {\emph{offers a 25\% tuition fee reduction}}
\end{entrylist}

\begin{aside}
  \section{Languages}
    \textbf{Chinese}\includegraphics[scale=0.40]{img/5stars.png}
    \textbf{English}\includegraphics[scale=0.40]{img/4stars.png}
\end{aside}

\section{Academy Congress and Conference}
On August 11, 2017, I took part in the Institute of Electrical and Electronics Engineers (IEEE) Sections Congress (SC2017), in Sydney, showing my final year project about Human Resource Management System based on Web.

\section{Extracurricular Activity}
I am a big fan of amateur radio (ham radio).
This is my biggest extracurricular activity.
In 2014, I have got my first amateur radio license (foundation license).
At that time, my purpose is talking and learning English because I can talk or send a message with all over the world radio amateur.
After my purpose is changed. I have found amateur radio have bigger relate with my major, computer science.
Amateur radio is not only sent morse code, which becomes history.
I am currently got my standard license.
So, I am more and more using new technology for radio.
Such as SSTV for the picture, D-STAR for digital mode voice transfer and data transfer.
Even International Space Station(ISS) also have got amateur radio FM repeater.
I am more and more focus on technique part for radio.
I also take part in some amateur radio competition.
For example, ross a. Hull memorial vhf-uhf contest.
I have got Twelfth Place in Section A (Analog Modes, Best 7 Days) and Twelfth Place in Section C (Analog Modes, Best 2 Days)).
This is only a part of my amateur radio activity.

\section{Research \& Project}
\subsection*{\textcolor{blue}{Software Engineering Aspect}}
\begin{entrylist}
  \entry
    {2018}
    {Testing in cross-language sentiment analysis\\}
    {University of Wollongong, Australia}
    {}
\end{entrylist}

\begin{entrylist}
  \entry
    {2018}
    {Metamorphic Testing in Text comparison tool\\}
    {University of Wollongong, Australia}
    {Using metamorphic testing method for testing DIFF utility quality. I have totally create 10 metamorphic relations.}
\end{entrylist}

\begin{entrylist}
  \entry
    {2017}
    {Testing in machine translation software\\}
    {University of Wollongong, Australia}
    {Using metamorphic testing method for testing and compare Google, Bing and Youdao machine translation software quality.\\\\ Github Link: \\{\small\url{https://github.com/yanboyang713/Metamorphic-Testing-of-Machine-Translation-Software.git}}}
\end{entrylist}

\begin{entrylist}
  \entry
    {2015}
    {UML in game shop\\}
    {University of Wollongong, Australia}
    {Using UML construct a use case diagram, class diagram, interaction diagram, context diagram and data flow diagram for a game shop.}
\end{entrylist}

\subsection*{\textcolor{blue}{Multimedia Aspect}}
\begin{entrylist}
  \entry
    {2017}
    {OPENGL construct a room\\}
    {University of Wollongong, Australia}
    {Using OPENGL construct a room with four walls and a floor, as well as two paintings, which are hung on separate walls, light, table and desk and so on.}
\end{entrylist}

\begin{entrylist}
  \entry
    {2017}
    {OPENGL planetary system\\}
    {University of Wollongong, Australia}
    {OpenGL program for a planetary system
    \\\\ Github Link: \\{\small\url{https://github.com/yanboyang713/openGLPlanetarySystem.git}}}
\end{entrylist}

\begin{entrylist}
  \entry
    {2016}
    {5th order low-pass Butterworth filter\\}
    {University of Wollongong, Australia}
    {Using SDL develop a program to process signal using the 5th order low-pass Butterworth filter and to play the processed signal.
     \\\\ Github Link: \\{\small\url{https://github.com/yanboyang713/butterworthFilter.git}}}
\end{entrylist}

\begin{entrylist}
  \entry
    {2016}
    {image histogram equalization\\}
    {University of Wollongong, Australia}
    {Using SDL develop a colour image display and enhancement program. The program enhances an image using histogram equalization and displays the enhanced image.
     \\\\ Github Link: \\{\small\url{https://github.com/yanboyang713/histogramEqualizationImage.git}}}
\end{entrylist}

\subsection{\textcolor{blue}{Algorithms Aspect}}

\begin{entrylist}
  \entry
    {2017}
    {Tries Data Structure\\}
    {University of Wollongong, Australia}
    {Doing some research about count unique word appear frequency and sort word frequency by decreasing count order. The main data structure is using tries to record each unique word.
     \\\\ Github Link: \\{\small\url{https://github.com/yanboyang713/tries-data-structure-count-unique-word-frequency.git}}}
\end{entrylist}

\begin{entrylist}
  \entry
    {2017}
    {simulate shop service\\}
    {University of Wollongong, Australia}
    {This project is about simulate shop service drive by event not by time.
     \\\\ Github Link: \\{\small\url{https://github.com/yanboyang713/simulate-shop-service.git}}}
\end{entrylist}

\subsection*{\textcolor{blue}{Security Aspect}}

\begin{entrylist}
  \entry
    {2017}
    {Rainbow Table\\}
    {University of Wollongong, Australia}
    {Implementing a rainbow table for Anti-Hash Function write by c++
    \\\\ Github Link: \\{\small\url{https://github.com/yanboyang713/rainbowTable.git}}}
\end{entrylist}

\subsection*{\textcolor{blue}{Hardware Aspect}}

\begin{entrylist}
  \entry
    {2017}
    {Swap Card System\\}
    {University of Wollongong, Australia}
    {Swap card system for record attendance. I have using Arduino, raspberry pi, the level shifter, LCD controller and LCD screen. In raspberry pi, I have written by Python, using HTTP socket connect with Java backend. And using I2C bus connect with Raspberry Pi, Arduino and LCD controller. Level shifter for convert different volt in the I2C bus. RFID sense connects with Arduino. In Arduino, I have written by c++.}
\end{entrylist}

\begin{entrylist}
  \entry
    {Working}
    {Google Home control light\\}
    {University of Wollongong, Australia}
    {Using NodeMCU with Google Home control light at home.}
\end{entrylist}

\subsection*{\textcolor{blue}{Database Aspect}}

\begin{entrylist}
  \entry
    {2017}
    {MYSQL Database\\}
    {University of Wollongong, Australia}
    {Using MYSQL doing human resource management system for final year project}
\end{entrylist}

\begin{entrylist}
  \entry
    {2017}
    {RDF Semantics Database\\}
    {University of Wollongong, Australia}
    {Using RDF semantics database for record movie data.}
\end{entrylist}

\subsection*{\textcolor{blue}{Web Aspect}}

\begin{entrylist}
  \entry
    {2016}
    {Blog\\}
    {University of Wollongong, Australia}
    {making my own blog using Jekyll.}
\end{entrylist}

\begin{entrylist}
  \entry
    {2014}
    {Website based on Drupal\\}
    {University of Wollongong, Australia}
    {Creating Illawarra film Society website based on Drupal.}
\end{entrylist}

\section{Referral Info}
\begin{itemize}
	\item George Zhou <zhiquan@uow.edu.au> Research Project
	
	\item Jack Yang <jiey@uow.edu.au> Research Project
	
	\item Eve Shaw <eve@uow.edu.au> English aspect
	
	\item Mark Freeman <mfreeman@uow.edu.au> undergraduate final project
	
	\item Casey Chow <caseyc@uow.edu.au> computer graphics aspect
	
	\item Tianbing Xia <txia@uow.edu.au> base programming aspect and database aspect
\end{itemize}


\begin{flushleft}
\emph{Oct 17th, 2018}
\end{flushleft}
\begin{flushright}
\emph{Boyang Yan}
\end{flushright}

%%% This piece of code has been commented by Karol Kozioł due to biblatex errors. 
% 
%\printbibsection{article}{article in peer-reviewed journal}
%\begin{refsection}
%  \nocite{*}
%  \printbibliography[sorting=chronological, type=inproceedings, title={international peer-reviewed conferences/proceedings}, notkeyword={france}, heading=subbibliography]
%\end{refsection}
%\begin{refsection}
%  \nocite{*}
%  \printbibliography[sorting=chronological, type=inproceedings, title={local peer-reviewed conferences/proceedings}, keyword={france}, heading=subbibliography]
%\end{refsection}
%\printbibsection{misc}{other publications}
%\printbibsection{report}{research reports}

\end{document}
